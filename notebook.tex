
% Default to the notebook output style

    


% Inherit from the specified cell style.




    
\documentclass[11pt]{article}

    
    
    \usepackage[T1]{fontenc}
    % Nicer default font (+ math font) than Computer Modern for most use cases
    \usepackage{mathpazo}

    % Basic figure setup, for now with no caption control since it's done
    % automatically by Pandoc (which extracts ![](path) syntax from Markdown).
    \usepackage{graphicx}
    % We will generate all images so they have a width \maxwidth. This means
    % that they will get their normal width if they fit onto the page, but
    % are scaled down if they would overflow the margins.
    \makeatletter
    \def\maxwidth{\ifdim\Gin@nat@width>\linewidth\linewidth
    \else\Gin@nat@width\fi}
    \makeatother
    \let\Oldincludegraphics\includegraphics
    % Set max figure width to be 80% of text width, for now hardcoded.
    \renewcommand{\includegraphics}[1]{\Oldincludegraphics[width=.8\maxwidth]{#1}}
    % Ensure that by default, figures have no caption (until we provide a
    % proper Figure object with a Caption API and a way to capture that
    % in the conversion process - todo).
    \usepackage{caption}
    \DeclareCaptionLabelFormat{nolabel}{}
    \captionsetup{labelformat=nolabel}

    \usepackage{adjustbox} % Used to constrain images to a maximum size 
    \usepackage{xcolor} % Allow colors to be defined
    \usepackage{enumerate} % Needed for markdown enumerations to work
    \usepackage{geometry} % Used to adjust the document margins
    \usepackage{amsmath} % Equations
    \usepackage{amssymb} % Equations
    \usepackage{textcomp} % defines textquotesingle
    % Hack from http://tex.stackexchange.com/a/47451/13684:
    \AtBeginDocument{%
        \def\PYZsq{\textquotesingle}% Upright quotes in Pygmentized code
    }
    \usepackage{upquote} % Upright quotes for verbatim code
    \usepackage{eurosym} % defines \euro
    \usepackage[mathletters]{ucs} % Extended unicode (utf-8) support
    \usepackage[utf8x]{inputenc} % Allow utf-8 characters in the tex document
    \usepackage{fancyvrb} % verbatim replacement that allows latex
    \usepackage{grffile} % extends the file name processing of package graphics 
                         % to support a larger range 
    % The hyperref package gives us a pdf with properly built
    % internal navigation ('pdf bookmarks' for the table of contents,
    % internal cross-reference links, web links for URLs, etc.)
    \usepackage{hyperref}
    \usepackage{longtable} % longtable support required by pandoc >1.10
    \usepackage{booktabs}  % table support for pandoc > 1.12.2
    \usepackage[inline]{enumitem} % IRkernel/repr support (it uses the enumerate* environment)
    \usepackage[normalem]{ulem} % ulem is needed to support strikethroughs (\sout)
                                % normalem makes italics be italics, not underlines
    

    
    
    % Colors for the hyperref package
    \definecolor{urlcolor}{rgb}{0,.145,.698}
    \definecolor{linkcolor}{rgb}{.71,0.21,0.01}
    \definecolor{citecolor}{rgb}{.12,.54,.11}

    % ANSI colors
    \definecolor{ansi-black}{HTML}{3E424D}
    \definecolor{ansi-black-intense}{HTML}{282C36}
    \definecolor{ansi-red}{HTML}{E75C58}
    \definecolor{ansi-red-intense}{HTML}{B22B31}
    \definecolor{ansi-green}{HTML}{00A250}
    \definecolor{ansi-green-intense}{HTML}{007427}
    \definecolor{ansi-yellow}{HTML}{DDB62B}
    \definecolor{ansi-yellow-intense}{HTML}{B27D12}
    \definecolor{ansi-blue}{HTML}{208FFB}
    \definecolor{ansi-blue-intense}{HTML}{0065CA}
    \definecolor{ansi-magenta}{HTML}{D160C4}
    \definecolor{ansi-magenta-intense}{HTML}{A03196}
    \definecolor{ansi-cyan}{HTML}{60C6C8}
    \definecolor{ansi-cyan-intense}{HTML}{258F8F}
    \definecolor{ansi-white}{HTML}{C5C1B4}
    \definecolor{ansi-white-intense}{HTML}{A1A6B2}

    % commands and environments needed by pandoc snippets
    % extracted from the output of `pandoc -s`
    \providecommand{\tightlist}{%
      \setlength{\itemsep}{0pt}\setlength{\parskip}{0pt}}
    \DefineVerbatimEnvironment{Highlighting}{Verbatim}{commandchars=\\\{\}}
    % Add ',fontsize=\small' for more characters per line
    \newenvironment{Shaded}{}{}
    \newcommand{\KeywordTok}[1]{\textcolor[rgb]{0.00,0.44,0.13}{\textbf{{#1}}}}
    \newcommand{\DataTypeTok}[1]{\textcolor[rgb]{0.56,0.13,0.00}{{#1}}}
    \newcommand{\DecValTok}[1]{\textcolor[rgb]{0.25,0.63,0.44}{{#1}}}
    \newcommand{\BaseNTok}[1]{\textcolor[rgb]{0.25,0.63,0.44}{{#1}}}
    \newcommand{\FloatTok}[1]{\textcolor[rgb]{0.25,0.63,0.44}{{#1}}}
    \newcommand{\CharTok}[1]{\textcolor[rgb]{0.25,0.44,0.63}{{#1}}}
    \newcommand{\StringTok}[1]{\textcolor[rgb]{0.25,0.44,0.63}{{#1}}}
    \newcommand{\CommentTok}[1]{\textcolor[rgb]{0.38,0.63,0.69}{\textit{{#1}}}}
    \newcommand{\OtherTok}[1]{\textcolor[rgb]{0.00,0.44,0.13}{{#1}}}
    \newcommand{\AlertTok}[1]{\textcolor[rgb]{1.00,0.00,0.00}{\textbf{{#1}}}}
    \newcommand{\FunctionTok}[1]{\textcolor[rgb]{0.02,0.16,0.49}{{#1}}}
    \newcommand{\RegionMarkerTok}[1]{{#1}}
    \newcommand{\ErrorTok}[1]{\textcolor[rgb]{1.00,0.00,0.00}{\textbf{{#1}}}}
    \newcommand{\NormalTok}[1]{{#1}}
    
    % Additional commands for more recent versions of Pandoc
    \newcommand{\ConstantTok}[1]{\textcolor[rgb]{0.53,0.00,0.00}{{#1}}}
    \newcommand{\SpecialCharTok}[1]{\textcolor[rgb]{0.25,0.44,0.63}{{#1}}}
    \newcommand{\VerbatimStringTok}[1]{\textcolor[rgb]{0.25,0.44,0.63}{{#1}}}
    \newcommand{\SpecialStringTok}[1]{\textcolor[rgb]{0.73,0.40,0.53}{{#1}}}
    \newcommand{\ImportTok}[1]{{#1}}
    \newcommand{\DocumentationTok}[1]{\textcolor[rgb]{0.73,0.13,0.13}{\textit{{#1}}}}
    \newcommand{\AnnotationTok}[1]{\textcolor[rgb]{0.38,0.63,0.69}{\textbf{\textit{{#1}}}}}
    \newcommand{\CommentVarTok}[1]{\textcolor[rgb]{0.38,0.63,0.69}{\textbf{\textit{{#1}}}}}
    \newcommand{\VariableTok}[1]{\textcolor[rgb]{0.10,0.09,0.49}{{#1}}}
    \newcommand{\ControlFlowTok}[1]{\textcolor[rgb]{0.00,0.44,0.13}{\textbf{{#1}}}}
    \newcommand{\OperatorTok}[1]{\textcolor[rgb]{0.40,0.40,0.40}{{#1}}}
    \newcommand{\BuiltInTok}[1]{{#1}}
    \newcommand{\ExtensionTok}[1]{{#1}}
    \newcommand{\PreprocessorTok}[1]{\textcolor[rgb]{0.74,0.48,0.00}{{#1}}}
    \newcommand{\AttributeTok}[1]{\textcolor[rgb]{0.49,0.56,0.16}{{#1}}}
    \newcommand{\InformationTok}[1]{\textcolor[rgb]{0.38,0.63,0.69}{\textbf{\textit{{#1}}}}}
    \newcommand{\WarningTok}[1]{\textcolor[rgb]{0.38,0.63,0.69}{\textbf{\textit{{#1}}}}}
    
    
    % Define a nice break command that doesn't care if a line doesn't already
    % exist.
    \def\br{\hspace*{\fill} \\* }
    % Math Jax compatability definitions
    \def\gt{>}
    \def\lt{<}
    % Document parameters
    \title{poco-quantico-com-barreiras}
    
    
    

    % Pygments definitions
    
\makeatletter
\def\PY@reset{\let\PY@it=\relax \let\PY@bf=\relax%
    \let\PY@ul=\relax \let\PY@tc=\relax%
    \let\PY@bc=\relax \let\PY@ff=\relax}
\def\PY@tok#1{\csname PY@tok@#1\endcsname}
\def\PY@toks#1+{\ifx\relax#1\empty\else%
    \PY@tok{#1}\expandafter\PY@toks\fi}
\def\PY@do#1{\PY@bc{\PY@tc{\PY@ul{%
    \PY@it{\PY@bf{\PY@ff{#1}}}}}}}
\def\PY#1#2{\PY@reset\PY@toks#1+\relax+\PY@do{#2}}

\expandafter\def\csname PY@tok@w\endcsname{\def\PY@tc##1{\textcolor[rgb]{0.73,0.73,0.73}{##1}}}
\expandafter\def\csname PY@tok@c\endcsname{\let\PY@it=\textit\def\PY@tc##1{\textcolor[rgb]{0.25,0.50,0.50}{##1}}}
\expandafter\def\csname PY@tok@cp\endcsname{\def\PY@tc##1{\textcolor[rgb]{0.74,0.48,0.00}{##1}}}
\expandafter\def\csname PY@tok@k\endcsname{\let\PY@bf=\textbf\def\PY@tc##1{\textcolor[rgb]{0.00,0.50,0.00}{##1}}}
\expandafter\def\csname PY@tok@kp\endcsname{\def\PY@tc##1{\textcolor[rgb]{0.00,0.50,0.00}{##1}}}
\expandafter\def\csname PY@tok@kt\endcsname{\def\PY@tc##1{\textcolor[rgb]{0.69,0.00,0.25}{##1}}}
\expandafter\def\csname PY@tok@o\endcsname{\def\PY@tc##1{\textcolor[rgb]{0.40,0.40,0.40}{##1}}}
\expandafter\def\csname PY@tok@ow\endcsname{\let\PY@bf=\textbf\def\PY@tc##1{\textcolor[rgb]{0.67,0.13,1.00}{##1}}}
\expandafter\def\csname PY@tok@nb\endcsname{\def\PY@tc##1{\textcolor[rgb]{0.00,0.50,0.00}{##1}}}
\expandafter\def\csname PY@tok@nf\endcsname{\def\PY@tc##1{\textcolor[rgb]{0.00,0.00,1.00}{##1}}}
\expandafter\def\csname PY@tok@nc\endcsname{\let\PY@bf=\textbf\def\PY@tc##1{\textcolor[rgb]{0.00,0.00,1.00}{##1}}}
\expandafter\def\csname PY@tok@nn\endcsname{\let\PY@bf=\textbf\def\PY@tc##1{\textcolor[rgb]{0.00,0.00,1.00}{##1}}}
\expandafter\def\csname PY@tok@ne\endcsname{\let\PY@bf=\textbf\def\PY@tc##1{\textcolor[rgb]{0.82,0.25,0.23}{##1}}}
\expandafter\def\csname PY@tok@nv\endcsname{\def\PY@tc##1{\textcolor[rgb]{0.10,0.09,0.49}{##1}}}
\expandafter\def\csname PY@tok@no\endcsname{\def\PY@tc##1{\textcolor[rgb]{0.53,0.00,0.00}{##1}}}
\expandafter\def\csname PY@tok@nl\endcsname{\def\PY@tc##1{\textcolor[rgb]{0.63,0.63,0.00}{##1}}}
\expandafter\def\csname PY@tok@ni\endcsname{\let\PY@bf=\textbf\def\PY@tc##1{\textcolor[rgb]{0.60,0.60,0.60}{##1}}}
\expandafter\def\csname PY@tok@na\endcsname{\def\PY@tc##1{\textcolor[rgb]{0.49,0.56,0.16}{##1}}}
\expandafter\def\csname PY@tok@nt\endcsname{\let\PY@bf=\textbf\def\PY@tc##1{\textcolor[rgb]{0.00,0.50,0.00}{##1}}}
\expandafter\def\csname PY@tok@nd\endcsname{\def\PY@tc##1{\textcolor[rgb]{0.67,0.13,1.00}{##1}}}
\expandafter\def\csname PY@tok@s\endcsname{\def\PY@tc##1{\textcolor[rgb]{0.73,0.13,0.13}{##1}}}
\expandafter\def\csname PY@tok@sd\endcsname{\let\PY@it=\textit\def\PY@tc##1{\textcolor[rgb]{0.73,0.13,0.13}{##1}}}
\expandafter\def\csname PY@tok@si\endcsname{\let\PY@bf=\textbf\def\PY@tc##1{\textcolor[rgb]{0.73,0.40,0.53}{##1}}}
\expandafter\def\csname PY@tok@se\endcsname{\let\PY@bf=\textbf\def\PY@tc##1{\textcolor[rgb]{0.73,0.40,0.13}{##1}}}
\expandafter\def\csname PY@tok@sr\endcsname{\def\PY@tc##1{\textcolor[rgb]{0.73,0.40,0.53}{##1}}}
\expandafter\def\csname PY@tok@ss\endcsname{\def\PY@tc##1{\textcolor[rgb]{0.10,0.09,0.49}{##1}}}
\expandafter\def\csname PY@tok@sx\endcsname{\def\PY@tc##1{\textcolor[rgb]{0.00,0.50,0.00}{##1}}}
\expandafter\def\csname PY@tok@m\endcsname{\def\PY@tc##1{\textcolor[rgb]{0.40,0.40,0.40}{##1}}}
\expandafter\def\csname PY@tok@gh\endcsname{\let\PY@bf=\textbf\def\PY@tc##1{\textcolor[rgb]{0.00,0.00,0.50}{##1}}}
\expandafter\def\csname PY@tok@gu\endcsname{\let\PY@bf=\textbf\def\PY@tc##1{\textcolor[rgb]{0.50,0.00,0.50}{##1}}}
\expandafter\def\csname PY@tok@gd\endcsname{\def\PY@tc##1{\textcolor[rgb]{0.63,0.00,0.00}{##1}}}
\expandafter\def\csname PY@tok@gi\endcsname{\def\PY@tc##1{\textcolor[rgb]{0.00,0.63,0.00}{##1}}}
\expandafter\def\csname PY@tok@gr\endcsname{\def\PY@tc##1{\textcolor[rgb]{1.00,0.00,0.00}{##1}}}
\expandafter\def\csname PY@tok@ge\endcsname{\let\PY@it=\textit}
\expandafter\def\csname PY@tok@gs\endcsname{\let\PY@bf=\textbf}
\expandafter\def\csname PY@tok@gp\endcsname{\let\PY@bf=\textbf\def\PY@tc##1{\textcolor[rgb]{0.00,0.00,0.50}{##1}}}
\expandafter\def\csname PY@tok@go\endcsname{\def\PY@tc##1{\textcolor[rgb]{0.53,0.53,0.53}{##1}}}
\expandafter\def\csname PY@tok@gt\endcsname{\def\PY@tc##1{\textcolor[rgb]{0.00,0.27,0.87}{##1}}}
\expandafter\def\csname PY@tok@err\endcsname{\def\PY@bc##1{\setlength{\fboxsep}{0pt}\fcolorbox[rgb]{1.00,0.00,0.00}{1,1,1}{\strut ##1}}}
\expandafter\def\csname PY@tok@kc\endcsname{\let\PY@bf=\textbf\def\PY@tc##1{\textcolor[rgb]{0.00,0.50,0.00}{##1}}}
\expandafter\def\csname PY@tok@kd\endcsname{\let\PY@bf=\textbf\def\PY@tc##1{\textcolor[rgb]{0.00,0.50,0.00}{##1}}}
\expandafter\def\csname PY@tok@kn\endcsname{\let\PY@bf=\textbf\def\PY@tc##1{\textcolor[rgb]{0.00,0.50,0.00}{##1}}}
\expandafter\def\csname PY@tok@kr\endcsname{\let\PY@bf=\textbf\def\PY@tc##1{\textcolor[rgb]{0.00,0.50,0.00}{##1}}}
\expandafter\def\csname PY@tok@bp\endcsname{\def\PY@tc##1{\textcolor[rgb]{0.00,0.50,0.00}{##1}}}
\expandafter\def\csname PY@tok@fm\endcsname{\def\PY@tc##1{\textcolor[rgb]{0.00,0.00,1.00}{##1}}}
\expandafter\def\csname PY@tok@vc\endcsname{\def\PY@tc##1{\textcolor[rgb]{0.10,0.09,0.49}{##1}}}
\expandafter\def\csname PY@tok@vg\endcsname{\def\PY@tc##1{\textcolor[rgb]{0.10,0.09,0.49}{##1}}}
\expandafter\def\csname PY@tok@vi\endcsname{\def\PY@tc##1{\textcolor[rgb]{0.10,0.09,0.49}{##1}}}
\expandafter\def\csname PY@tok@vm\endcsname{\def\PY@tc##1{\textcolor[rgb]{0.10,0.09,0.49}{##1}}}
\expandafter\def\csname PY@tok@sa\endcsname{\def\PY@tc##1{\textcolor[rgb]{0.73,0.13,0.13}{##1}}}
\expandafter\def\csname PY@tok@sb\endcsname{\def\PY@tc##1{\textcolor[rgb]{0.73,0.13,0.13}{##1}}}
\expandafter\def\csname PY@tok@sc\endcsname{\def\PY@tc##1{\textcolor[rgb]{0.73,0.13,0.13}{##1}}}
\expandafter\def\csname PY@tok@dl\endcsname{\def\PY@tc##1{\textcolor[rgb]{0.73,0.13,0.13}{##1}}}
\expandafter\def\csname PY@tok@s2\endcsname{\def\PY@tc##1{\textcolor[rgb]{0.73,0.13,0.13}{##1}}}
\expandafter\def\csname PY@tok@sh\endcsname{\def\PY@tc##1{\textcolor[rgb]{0.73,0.13,0.13}{##1}}}
\expandafter\def\csname PY@tok@s1\endcsname{\def\PY@tc##1{\textcolor[rgb]{0.73,0.13,0.13}{##1}}}
\expandafter\def\csname PY@tok@mb\endcsname{\def\PY@tc##1{\textcolor[rgb]{0.40,0.40,0.40}{##1}}}
\expandafter\def\csname PY@tok@mf\endcsname{\def\PY@tc##1{\textcolor[rgb]{0.40,0.40,0.40}{##1}}}
\expandafter\def\csname PY@tok@mh\endcsname{\def\PY@tc##1{\textcolor[rgb]{0.40,0.40,0.40}{##1}}}
\expandafter\def\csname PY@tok@mi\endcsname{\def\PY@tc##1{\textcolor[rgb]{0.40,0.40,0.40}{##1}}}
\expandafter\def\csname PY@tok@il\endcsname{\def\PY@tc##1{\textcolor[rgb]{0.40,0.40,0.40}{##1}}}
\expandafter\def\csname PY@tok@mo\endcsname{\def\PY@tc##1{\textcolor[rgb]{0.40,0.40,0.40}{##1}}}
\expandafter\def\csname PY@tok@ch\endcsname{\let\PY@it=\textit\def\PY@tc##1{\textcolor[rgb]{0.25,0.50,0.50}{##1}}}
\expandafter\def\csname PY@tok@cm\endcsname{\let\PY@it=\textit\def\PY@tc##1{\textcolor[rgb]{0.25,0.50,0.50}{##1}}}
\expandafter\def\csname PY@tok@cpf\endcsname{\let\PY@it=\textit\def\PY@tc##1{\textcolor[rgb]{0.25,0.50,0.50}{##1}}}
\expandafter\def\csname PY@tok@c1\endcsname{\let\PY@it=\textit\def\PY@tc##1{\textcolor[rgb]{0.25,0.50,0.50}{##1}}}
\expandafter\def\csname PY@tok@cs\endcsname{\let\PY@it=\textit\def\PY@tc##1{\textcolor[rgb]{0.25,0.50,0.50}{##1}}}

\def\PYZbs{\char`\\}
\def\PYZus{\char`\_}
\def\PYZob{\char`\{}
\def\PYZcb{\char`\}}
\def\PYZca{\char`\^}
\def\PYZam{\char`\&}
\def\PYZlt{\char`\<}
\def\PYZgt{\char`\>}
\def\PYZsh{\char`\#}
\def\PYZpc{\char`\%}
\def\PYZdl{\char`\$}
\def\PYZhy{\char`\-}
\def\PYZsq{\char`\'}
\def\PYZdq{\char`\"}
\def\PYZti{\char`\~}
% for compatibility with earlier versions
\def\PYZat{@}
\def\PYZlb{[}
\def\PYZrb{]}
\makeatother


    % Exact colors from NB
    \definecolor{incolor}{rgb}{0.0, 0.0, 0.5}
    \definecolor{outcolor}{rgb}{0.545, 0.0, 0.0}



    
    % Prevent overflowing lines due to hard-to-break entities
    \sloppy 
    % Setup hyperref package
    \hypersetup{
      breaklinks=true,  % so long urls are correctly broken across lines
      colorlinks=true,
      urlcolor=urlcolor,
      linkcolor=linkcolor,
      citecolor=citecolor,
      }
    % Slightly bigger margins than the latex defaults
    
    \geometry{verbose,tmargin=1in,bmargin=1in,lmargin=1in,rmargin=1in}
    
    

    \begin{document}
    
    
    \maketitle
    
    

    
    \hypertarget{estudo-da-fotocorrente-absoruxe7uxe3o-reflexuxe3o-e-transmissuxe3o-em-heteroestruturas-semicondutora}{%
\section{Estudo da Fotocorrente, Absorção, Reflexão e Transmissão em
Heteroestruturas
Semicondutora}\label{estudo-da-fotocorrente-absoruxe7uxe3o-reflexuxe3o-e-transmissuxe3o-em-heteroestruturas-semicondutora}}

Aqui vamos analizar a heteroestrutura proposta por Degani et al {[}1{]}.

\hypertarget{bibliotecas-utilizadas}{%
\subsection{Bibliotecas utilizadas}\label{bibliotecas-utilizadas}}

    \begin{Verbatim}[commandchars=\\\{\}]
{\color{incolor}In [{\color{incolor}1}]:} \PY{k+kn}{import} \PY{n+nn}{os}
        \PY{k+kn}{import} \PY{n+nn}{time}
        \PY{k+kn}{import} \PY{n+nn}{re}
        \PY{k+kn}{from} \PY{n+nn}{multiprocessing} \PY{k}{import} \PY{n}{Pool}\PY{p}{,} \PY{n+ne}{TimeoutError}
        \PY{k+kn}{from} \PY{n+nn}{datetime} \PY{k}{import} \PY{n}{datetime}
        \PY{k+kn}{import} \PY{n+nn}{numpy} \PY{k}{as} \PY{n+nn}{np}
        \PY{k+kn}{import} \PY{n+nn}{pandas} \PY{k}{as} \PY{n+nn}{pd}
        \PY{k+kn}{from} \PY{n+nn}{scipy} \PY{k}{import} \PY{n}{constants} \PY{k}{as} \PY{n}{cte}
        \PY{k+kn}{from} \PY{n+nn}{scipy}\PY{n+nn}{.}\PY{n+nn}{fftpack} \PY{k}{import} \PY{n}{fft}\PY{p}{,} \PY{n}{ifft}\PY{p}{,} \PY{n}{fftfreq}
        \PY{k+kn}{from} \PY{n+nn}{scipy}\PY{n+nn}{.}\PY{n+nn}{special} \PY{k}{import} \PY{n}{expit}
        \PY{k+kn}{from} \PY{n+nn}{scipy} \PY{k}{import} \PY{n}{constants} \PY{k}{as} \PY{n}{cte}
        \PY{k+kn}{from} \PY{n+nn}{scipy}\PY{n+nn}{.}\PY{n+nn}{integrate} \PY{k}{import} \PY{n}{simps}
        \PY{k+kn}{from} \PY{n+nn}{scipy}\PY{n+nn}{.}\PY{n+nn}{sparse} \PY{k}{import} \PY{n}{diags}
        \PY{k+kn}{from} \PY{n+nn}{scipy}\PY{n+nn}{.}\PY{n+nn}{linalg} \PY{k}{import} \PY{n}{inv}
        \PY{k+kn}{from} \PY{n+nn}{scipy}\PY{n+nn}{.}\PY{n+nn}{signal} \PY{k}{import} \PY{n}{gaussian}
        \PY{k+kn}{from} \PY{n+nn}{scipy}\PY{n+nn}{.}\PY{n+nn}{special} \PY{k}{import} \PY{n}{legendre}\PY{p}{,} \PY{n}{expit}
        \PY{k+kn}{from} \PY{n+nn}{scipy}\PY{n+nn}{.}\PY{n+nn}{fftpack} \PY{k}{import} \PY{n}{fft}\PY{p}{,} \PY{n}{ifft}\PY{p}{,} \PY{n}{fftfreq}
        \PY{k+kn}{from} \PY{n+nn}{scipy}\PY{n+nn}{.}\PY{n+nn}{spatial}\PY{n+nn}{.}\PY{n+nn}{distance} \PY{k}{import} \PY{n}{cdist}
\end{Verbatim}


    \hypertarget{configurauxe7uxe3o-estuxe9tica}{%
\subsection{Configuração
Estética}\label{configurauxe7uxe3o-estuxe9tica}}

    \begin{Verbatim}[commandchars=\\\{\}]
{\color{incolor}In [{\color{incolor}5}]:} \PY{o}{\PYZpc{}}\PY{k}{matplotlib} inline
        \PY{k+kn}{import} \PY{n+nn}{matplotlib}\PY{n+nn}{.}\PY{n+nn}{pyplot} \PY{k}{as} \PY{n+nn}{plt}
        \PY{k+kn}{from} \PY{n+nn}{matplotlib}\PY{n+nn}{.}\PY{n+nn}{ticker} \PY{k}{import} \PY{n}{MultipleLocator}
        \PY{k+kn}{from} \PY{n+nn}{IPython}\PY{n+nn}{.}\PY{n+nn}{display} \PY{k}{import} \PY{n}{set\PYZus{}matplotlib\PYZus{}formats}
        \PY{n}{set\PYZus{}matplotlib\PYZus{}formats}\PY{p}{(}\PY{l+s+s1}{\PYZsq{}}\PY{l+s+s1}{pdf}\PY{l+s+s1}{\PYZsq{}}\PY{p}{,} \PY{l+s+s1}{\PYZsq{}}\PY{l+s+s1}{png}\PY{l+s+s1}{\PYZsq{}}\PY{p}{)}
        \PY{n}{plt}\PY{o}{.}\PY{n}{rcParams}\PY{p}{[}\PY{l+s+s1}{\PYZsq{}}\PY{l+s+s1}{savefig.dpi}\PY{l+s+s1}{\PYZsq{}}\PY{p}{]} \PY{o}{=} \PY{l+m+mi}{300}
        \PY{n}{plt}\PY{o}{.}\PY{n}{rcParams}\PY{p}{[}\PY{l+s+s1}{\PYZsq{}}\PY{l+s+s1}{figure.autolayout}\PY{l+s+s1}{\PYZsq{}}\PY{p}{]} \PY{o}{=} \PY{k+kc}{False}
        \PY{n}{plt}\PY{o}{.}\PY{n}{rcParams}\PY{p}{[}\PY{l+s+s1}{\PYZsq{}}\PY{l+s+s1}{figure.figsize}\PY{l+s+s1}{\PYZsq{}}\PY{p}{]} \PY{o}{=} \PY{l+m+mf}{11.7}\PY{p}{,} \PY{l+m+mf}{8.27}
        \PY{n}{plt}\PY{o}{.}\PY{n}{rcParams}\PY{p}{[}\PY{l+s+s1}{\PYZsq{}}\PY{l+s+s1}{axes.labelsize}\PY{l+s+s1}{\PYZsq{}}\PY{p}{]} \PY{o}{=} \PY{l+m+mi}{18}
        \PY{n}{plt}\PY{o}{.}\PY{n}{rcParams}\PY{p}{[}\PY{l+s+s1}{\PYZsq{}}\PY{l+s+s1}{axes.titlesize}\PY{l+s+s1}{\PYZsq{}}\PY{p}{]} \PY{o}{=} \PY{l+m+mi}{20}
        \PY{n}{plt}\PY{o}{.}\PY{n}{rcParams}\PY{p}{[}\PY{l+s+s1}{\PYZsq{}}\PY{l+s+s1}{font.size}\PY{l+s+s1}{\PYZsq{}}\PY{p}{]} \PY{o}{=} \PY{l+m+mi}{16}
        \PY{n}{plt}\PY{o}{.}\PY{n}{rcParams}\PY{p}{[}\PY{l+s+s1}{\PYZsq{}}\PY{l+s+s1}{lines.linewidth}\PY{l+s+s1}{\PYZsq{}}\PY{p}{]} \PY{o}{=} \PY{l+m+mf}{2.0}
        \PY{n}{plt}\PY{o}{.}\PY{n}{rcParams}\PY{p}{[}\PY{l+s+s1}{\PYZsq{}}\PY{l+s+s1}{lines.markersize}\PY{l+s+s1}{\PYZsq{}}\PY{p}{]} \PY{o}{=} \PY{l+m+mi}{8}
        \PY{n}{plt}\PY{o}{.}\PY{n}{rcParams}\PY{p}{[}\PY{l+s+s1}{\PYZsq{}}\PY{l+s+s1}{legend.fontsize}\PY{l+s+s1}{\PYZsq{}}\PY{p}{]} \PY{o}{=} \PY{l+m+mi}{16}
        \PY{c+c1}{\PYZsh{} plt.rcParams[\PYZsq{}text.usetex\PYZsq{}] = True}
        \PY{c+c1}{\PYZsh{} plt.rcParams[\PYZsq{}font.family\PYZsq{}] = \PYZdq{}serif\PYZdq{}}
        \PY{c+c1}{\PYZsh{} plt.rcParams[\PYZsq{}font.serif\PYZsq{}] = \PYZdq{}cm\PYZdq{}}
\end{Verbatim}


    \hypertarget{constantes-e-definiuxe7uxf5es}{%
\subsection{Constantes e
Definições}\label{constantes-e-definiuxe7uxf5es}}

    \begin{Verbatim}[commandchars=\\\{\}]
{\color{incolor}In [{\color{incolor}3}]:} \PY{c+c1}{\PYZsh{} dataframe de pandas com valores utilizados para calculos}
        \PY{n}{device} \PY{o}{=} \PY{n}{pd}\PY{o}{.}\PY{n}{DataFrame}\PY{p}{(}\PY{p}{)}
        
        \PY{n}{N} \PY{o}{=} \PY{l+m+mi}{1024}  \PY{c+c1}{\PYZsh{} tamanho padrao do grid}
        \PY{n}{L} \PY{o}{=} \PY{l+m+mf}{1000.0}  \PY{c+c1}{\PYZsh{} tamanho padrao do sistema em angstrom}
        \PY{n}{dt} \PY{o}{=} \PY{l+m+mf}{1e\PYZhy{}17}  \PY{c+c1}{\PYZsh{} incremento de tempo padrao em segundos}
        \PY{n}{device}\PY{p}{[}\PY{l+s+s1}{\PYZsq{}}\PY{l+s+s1}{z\PYZus{}ang}\PY{l+s+s1}{\PYZsq{}}\PY{p}{]} \PY{o}{=} \PY{n}{np}\PY{o}{.}\PY{n}{linspace}\PY{p}{(}\PY{o}{\PYZhy{}}\PY{n}{L}\PY{o}{/}\PY{l+m+mi}{2}\PY{p}{,} \PY{n}{L}\PY{o}{/}\PY{l+m+mi}{2}\PY{p}{,} \PY{n}{N}\PY{p}{)}  \PY{c+c1}{\PYZsh{} malha espacial em angstrom}
        
        \PY{c+c1}{\PYZsh{} fatores de conversao para unidades atomicas}
        \PY{n}{au\PYZus{}l} \PY{o}{=} \PY{n}{cte}\PY{o}{.}\PY{n}{value}\PY{p}{(}\PY{l+s+s1}{\PYZsq{}}\PY{l+s+s1}{atomic unit of length}\PY{l+s+s1}{\PYZsq{}}\PY{p}{)}
        \PY{n}{au\PYZus{}t} \PY{o}{=} \PY{n}{cte}\PY{o}{.}\PY{n}{value}\PY{p}{(}\PY{l+s+s1}{\PYZsq{}}\PY{l+s+s1}{atomic unit of time}\PY{l+s+s1}{\PYZsq{}}\PY{p}{)}
        \PY{n}{au\PYZus{}e} \PY{o}{=} \PY{n}{cte}\PY{o}{.}\PY{n}{value}\PY{p}{(}\PY{l+s+s1}{\PYZsq{}}\PY{l+s+s1}{atomic unit of energy}\PY{l+s+s1}{\PYZsq{}}\PY{p}{)}
        \PY{n}{au\PYZus{}v} \PY{o}{=} \PY{n}{cte}\PY{o}{.}\PY{n}{value}\PY{p}{(}\PY{l+s+s1}{\PYZsq{}}\PY{l+s+s1}{atomic unit of electric potential}\PY{l+s+s1}{\PYZsq{}}\PY{p}{)}
        \PY{n}{hbar\PYZus{}au} \PY{o}{=} \PY{l+m+mf}{1.0}
        \PY{n}{me\PYZus{}au} \PY{o}{=} \PY{l+m+mf}{1.0}
        
        \PY{c+c1}{\PYZsh{} constantes fisicas}
        \PY{n}{ev} \PY{o}{=} \PY{n}{cte}\PY{o}{.}\PY{n}{value}\PY{p}{(}\PY{l+s+s1}{\PYZsq{}}\PY{l+s+s1}{electron volt}\PY{l+s+s1}{\PYZsq{}}\PY{p}{)}
        \PY{n}{c} \PY{o}{=} \PY{n}{cte}\PY{o}{.}\PY{n}{value}\PY{p}{(}\PY{l+s+s1}{\PYZsq{}}\PY{l+s+s1}{speed of light in vacuum}\PY{l+s+s1}{\PYZsq{}}\PY{p}{)}
        \PY{n}{me} \PY{o}{=} \PY{n}{cte}\PY{o}{.}\PY{n}{value}\PY{p}{(}\PY{l+s+s1}{\PYZsq{}}\PY{l+s+s1}{electron mass}\PY{l+s+s1}{\PYZsq{}}\PY{p}{)}
        \PY{n}{q} \PY{o}{=} \PY{n}{cte}\PY{o}{.}\PY{n}{value}\PY{p}{(}\PY{l+s+s1}{\PYZsq{}}\PY{l+s+s1}{elementary charge}\PY{l+s+s1}{\PYZsq{}}\PY{p}{)}
        
        \PY{c+c1}{\PYZsh{} conversoes rapidas}
        \PY{n}{au2ev} \PY{o}{=} \PY{n}{au\PYZus{}e} \PY{o}{/} \PY{n}{ev}  \PY{c+c1}{\PYZsh{} unidades atomicas para eV}
        \PY{n}{au2ang} \PY{o}{=} \PY{n}{au\PYZus{}l} \PY{o}{/} \PY{l+m+mf}{1e\PYZhy{}10}  \PY{c+c1}{\PYZsh{} unidades atomicas para angstrom}
        
        \PY{k}{def} \PY{n+nf}{algaas\PYZus{}gap}\PY{p}{(}\PY{n}{x}\PY{p}{:} \PY{n+nb}{float}\PY{p}{)} \PY{o}{\PYZhy{}}\PY{o}{\PYZgt{}} \PY{n+nb}{float}\PY{p}{:}
            \PY{l+s+sd}{\PYZdq{}\PYZdq{}\PYZdq{}Retorna o gap do material ja calculado em funcao da fracao de Aluminio}
        \PY{l+s+sd}{    utilizamos os valores exatos utilizados pelos referidos autores}
        
        \PY{l+s+sd}{    :param x: a fracao de aluminio, entre 0 e 1}
        \PY{l+s+sd}{    :returns: o gap em eV}
        \PY{l+s+sd}{    \PYZdq{}\PYZdq{}\PYZdq{}}
            \PY{k}{if} \PY{n}{x} \PY{o}{==} \PY{l+m+mf}{0.2}\PY{p}{:}
                \PY{k}{return} \PY{l+m+mf}{0.0}
            \PY{k}{elif} \PY{n}{x} \PY{o}{==} \PY{l+m+mf}{0.4}\PY{p}{:}
                \PY{k}{return} \PY{l+m+mf}{0.185897}
            \PY{k}{return} \PY{o}{\PYZhy{}}\PY{l+m+mf}{0.185897}
        
        
        \PY{k}{def} \PY{n+nf}{algaas\PYZus{}meff}\PY{p}{(}\PY{n}{x}\PY{p}{:} \PY{n+nb}{float}\PY{p}{)} \PY{o}{\PYZhy{}}\PY{o}{\PYZgt{}} \PY{n+nb}{float}\PY{p}{:}
            \PY{l+s+sd}{\PYZdq{}\PYZdq{}\PYZdq{}Retorna a massa efetiva do AlGaAs em funcao da fracao de Aluminio}
        \PY{l+s+sd}{    assim como os referidos autores, utilizamos a massa efetiva do }
        \PY{l+s+sd}{    eletron no GaAs ao longo de todo o material}
        
        \PY{l+s+sd}{    :param x: a fracao de aluminio, entre 0 e 1}
        \PY{l+s+sd}{    :returns: a massa efetiva do eletron no AlGaAs}
        \PY{l+s+sd}{    \PYZdq{}\PYZdq{}\PYZdq{}}
            \PY{k}{return} \PY{l+m+mf}{0.067}
\end{Verbatim}


    \hypertarget{perfil-do-potencial}{%
\subsection{Perfil do Potencial}\label{perfil-do-potencial}}

    \begin{Verbatim}[commandchars=\\\{\}]
{\color{incolor}In [{\color{incolor}11}]:} \PY{k}{def} \PY{n+nf}{x\PYZus{}shape}\PY{p}{(}\PY{n}{z}\PY{p}{:} \PY{n+nb}{float}\PY{p}{)} \PY{o}{\PYZhy{}}\PY{o}{\PYZgt{}} \PY{n+nb}{float}\PY{p}{:}
             \PY{l+s+sd}{\PYZdq{}\PYZdq{}\PYZdq{}Utilizamos a concentracao de Aluminio para determinar o perfil do}
         \PY{l+s+sd}{    potencial}
         \PY{l+s+sd}{    }
         \PY{l+s+sd}{    :param z: posicao no eixo z em angstrom}
         \PY{l+s+sd}{    :return: a concentracao de Aluminio na posicao informada}
         \PY{l+s+sd}{    \PYZdq{}\PYZdq{}\PYZdq{}}
             \PY{c+c1}{\PYZsh{} concentracoes e larguras do sistema}
             \PY{n}{xd} \PY{o}{=} \PY{l+m+mf}{0.2}  \PY{c+c1}{\PYZsh{} concentracao no espaco entre poco e barreira}
             \PY{n}{xb} \PY{o}{=} \PY{l+m+mf}{0.4}  \PY{c+c1}{\PYZsh{} concentracao na barreira}
             \PY{n}{xw} \PY{o}{=} \PY{l+m+mf}{0.0}  \PY{c+c1}{\PYZsh{} concentracao no poco}
             \PY{n}{wl} \PY{o}{=} \PY{l+m+mf}{50.0}  \PY{c+c1}{\PYZsh{} largura do poco em angstrom}
             \PY{n}{bl} \PY{o}{=} \PY{l+m+mf}{50.0}  \PY{c+c1}{\PYZsh{} largura da barreira em angstrom}
             \PY{n}{dl} \PY{o}{=} \PY{l+m+mf}{40.0}  \PY{c+c1}{\PYZsh{} espacao entre poco e barreira em angstrom}
             
             \PY{k}{if} \PY{n}{np}\PY{o}{.}\PY{n}{abs}\PY{p}{(}\PY{n}{z}\PY{p}{)} \PY{o}{\PYZlt{}} \PY{n}{wl}\PY{o}{/}\PY{l+m+mi}{2}\PY{p}{:}
                 \PY{k}{return} \PY{n}{xw}
             \PY{k}{elif} \PY{n}{np}\PY{o}{.}\PY{n}{abs}\PY{p}{(}\PY{n}{z}\PY{p}{)} \PY{o}{\PYZlt{}} \PY{n}{wl}\PY{o}{/}\PY{l+m+mi}{2}\PY{o}{+}\PY{n}{dl}\PY{p}{:}
                 \PY{k}{return} \PY{n}{xd}
             \PY{k}{elif} \PY{n}{np}\PY{o}{.}\PY{n}{abs}\PY{p}{(}\PY{n}{z}\PY{p}{)} \PY{o}{\PYZlt{}} \PY{n}{wl}\PY{o}{/}\PY{l+m+mi}{2}\PY{o}{+}\PY{n}{dl}\PY{o}{+}\PY{n}{bl}\PY{p}{:}
                 \PY{k}{return} \PY{n}{xb}
             \PY{k}{return} \PY{n}{xd}
         
         
         \PY{n}{device}\PY{p}{[}\PY{l+s+s1}{\PYZsq{}}\PY{l+s+s1}{x}\PY{l+s+s1}{\PYZsq{}}\PY{p}{]} \PY{o}{=} \PY{n}{device}\PY{p}{[}\PY{l+s+s1}{\PYZsq{}}\PY{l+s+s1}{z\PYZus{}ang}\PY{l+s+s1}{\PYZsq{}}\PY{p}{]}\PY{o}{.}\PY{n}{apply}\PY{p}{(}\PY{n}{x\PYZus{}shape}\PY{p}{)}
         \PY{n}{device}\PY{p}{[}\PY{l+s+s1}{\PYZsq{}}\PY{l+s+s1}{v\PYZus{}ev}\PY{l+s+s1}{\PYZsq{}}\PY{p}{]} \PY{o}{=} \PY{n}{device}\PY{p}{[}\PY{l+s+s1}{\PYZsq{}}\PY{l+s+s1}{x}\PY{l+s+s1}{\PYZsq{}}\PY{p}{]}\PY{o}{.}\PY{n}{apply}\PY{p}{(}\PY{n}{algaas\PYZus{}gap}\PY{p}{)}
         \PY{n}{device}\PY{p}{[}\PY{l+s+s1}{\PYZsq{}}\PY{l+s+s1}{meff}\PY{l+s+s1}{\PYZsq{}}\PY{p}{]} \PY{o}{=} \PY{n}{device}\PY{p}{[}\PY{l+s+s1}{\PYZsq{}}\PY{l+s+s1}{x}\PY{l+s+s1}{\PYZsq{}}\PY{p}{]}\PY{o}{.}\PY{n}{apply}\PY{p}{(}\PY{n}{algaas\PYZus{}meff}\PY{p}{)}
         
         \PY{n}{dc} \PY{o}{=} \PY{n}{device}\PY{o}{.}\PY{n}{plot}\PY{p}{(}\PY{n}{x}\PY{o}{=}\PY{l+s+s1}{\PYZsq{}}\PY{l+s+s1}{z\PYZus{}ang}\PY{l+s+s1}{\PYZsq{}}\PY{p}{,} \PY{n}{y}\PY{o}{=}\PY{l+s+s1}{\PYZsq{}}\PY{l+s+s1}{v\PYZus{}ev}\PY{l+s+s1}{\PYZsq{}}\PY{p}{,} \PY{n}{grid}\PY{o}{=}\PY{k+kc}{True}\PY{p}{,} \PY{n}{title}\PY{o}{=}\PY{l+s+s1}{\PYZsq{}}\PY{l+s+s1}{V(eV) x z (Ang)}\PY{l+s+s1}{\PYZsq{}}\PY{p}{)}
\end{Verbatim}


    \begin{center}
    \adjustimage{max size={0.9\linewidth}{0.9\paperheight}}{output_7_0.pdf}
    \end{center}
    { \hspace*{\fill} \\}
    
    \hypertarget{aplicando-campo-estuxe1tico}{%
\subsection{Aplicando campo
estático}\label{aplicando-campo-estuxe1tico}}

    \begin{Verbatim}[commandchars=\\\{\}]
{\color{incolor}In [{\color{incolor}16}]:} \PY{n}{bias} \PY{o}{=} \PY{l+m+mf}{5.0}  \PY{c+c1}{\PYZsh{} KV/cm}
         \PY{n}{bias\PYZus{}v\PYZus{}cm} \PY{o}{=} \PY{n}{bias} \PY{o}{*} \PY{l+m+mf}{1e3}
         \PY{n}{bias\PYZus{}v\PYZus{}m} \PY{o}{=} \PY{l+m+mf}{1e2} \PY{o}{*} \PY{n}{bias\PYZus{}v\PYZus{}cm}
         \PY{n}{bias\PYZus{}j\PYZus{}m} \PY{o}{=} \PY{n}{bias\PYZus{}v\PYZus{}m} \PY{o}{*} \PY{n}{q}
         
         \PY{k}{def} \PY{n+nf}{derivada\PYZus{}muda\PYZus{}pct}\PY{p}{(}\PY{n}{x}\PY{p}{:} \PY{n+nb}{list}\PY{p}{,} \PY{n}{y}\PY{p}{:} \PY{n+nb}{list}\PY{p}{,} \PY{n}{n}\PY{p}{:} \PY{n+nb}{int} \PY{o}{=} \PY{l+m+mi}{10}\PY{p}{,} \PY{n}{pct}\PY{p}{:} \PY{n+nb}{float} \PY{o}{=} \PY{l+m+mf}{0.05}\PY{p}{)} \PY{o}{\PYZhy{}}\PY{o}{\PYZgt{}} \PY{n+nb}{int}\PY{p}{:}
             \PY{l+s+sd}{\PYZdq{}\PYZdq{}\PYZdq{}encontra o ponto x onde a derivada de y(x) muda mais do que uma }
         \PY{l+s+sd}{    certa porcentagem pela primeira vez, da esquerda para a direita}
         
         \PY{l+s+sd}{    :param x: um array com os valores em x}
         \PY{l+s+sd}{    :param y: um array com os valores em y}
         \PY{l+s+sd}{    :param n: numero do pontos para ignorar nas bordas}
         \PY{l+s+sd}{    :param pct: a porcentagem da derivada que deve mudar}
         \PY{l+s+sd}{    :returns: o indice do ponto x onde dy/dx muda mais do que pct}
         \PY{l+s+sd}{    \PYZdq{}\PYZdq{}\PYZdq{}}
             \PY{n}{der\PYZus{}y} \PY{o}{=} \PY{n}{np}\PY{o}{.}\PY{n}{array}\PY{p}{(}\PY{n}{y}\PY{p}{[}\PY{l+m+mi}{2}\PY{p}{:}\PY{p}{]}\PY{o}{\PYZhy{}}\PY{n}{y}\PY{p}{[}\PY{p}{:}\PY{o}{\PYZhy{}}\PY{l+m+mi}{2}\PY{p}{]}\PY{p}{)}\PY{o}{/}\PY{n}{np}\PY{o}{.}\PY{n}{array}\PY{p}{(}\PY{n}{x}\PY{p}{[}\PY{l+m+mi}{2}\PY{p}{:}\PY{p}{]}\PY{o}{\PYZhy{}}\PY{n}{x}\PY{p}{[}\PY{p}{:}\PY{o}{\PYZhy{}}\PY{l+m+mi}{2}\PY{p}{]}\PY{p}{)}
             \PY{k}{for} \PY{n}{i} \PY{o+ow}{in} \PY{n+nb}{range}\PY{p}{(}\PY{n}{n}\PY{p}{,} \PY{n+nb}{len}\PY{p}{(}\PY{n}{der\PYZus{}y}\PY{p}{)}\PY{p}{)}\PY{p}{:}
                 \PY{n}{last\PYZus{}n} \PY{o}{=} \PY{n}{np}\PY{o}{.}\PY{n}{average}\PY{p}{(}\PY{n}{der\PYZus{}y}\PY{p}{[}\PY{n}{i}\PY{o}{\PYZhy{}}\PY{n}{n}\PY{p}{:}\PY{n}{i}\PY{o}{\PYZhy{}}\PY{l+m+mi}{1}\PY{p}{]}\PY{p}{)}
                 \PY{k}{if} \PY{n}{last\PYZus{}n} \PY{o}{==} \PY{l+m+mi}{0} \PY{o+ow}{and} \PY{n}{der\PYZus{}y}\PY{p}{[}\PY{n}{i}\PY{p}{]} \PY{o}{!=} \PY{l+m+mi}{0} \PYZbs{}
                         \PY{o+ow}{or} \PY{n}{last\PYZus{}n} \PY{o}{!=} \PY{l+m+mi}{0} \PY{o+ow}{and} \PY{n}{np}\PY{o}{.}\PY{n}{abs}\PY{p}{(}\PY{n}{der\PYZus{}y}\PY{p}{[}\PY{n}{i}\PY{p}{]}\PY{o}{/}\PY{n}{last\PYZus{}n}\PY{o}{\PYZhy{}}\PY{l+m+mi}{1}\PY{p}{)} \PY{o}{\PYZgt{}} \PY{n}{pct}\PY{p}{:}
                     \PY{k}{return} \PY{n}{i}
             \PY{k}{return} \PY{n+nb}{int}\PY{p}{(}\PY{n+nb}{len}\PY{p}{(}\PY{n}{y}\PY{p}{)}\PY{o}{/}\PY{l+m+mi}{3}\PY{p}{)}
         
         \PY{k}{def} \PY{n+nf}{constroi\PYZus{}bias}\PY{p}{(}\PY{n}{z}\PY{p}{:} \PY{n+nb}{float}\PY{p}{)} \PY{o}{\PYZhy{}}\PY{o}{\PYZgt{}} \PY{n+nb}{float}\PY{p}{:}
             \PY{l+s+sd}{\PYZdq{}\PYZdq{}\PYZdq{}constroi o potencial estatico usado como bias/vies}
         \PY{l+s+sd}{    }
         \PY{l+s+sd}{    :param z: uma posicao no grid em angstrom}
         \PY{l+s+sd}{    :returns: o bias na posicao indicada}
         \PY{l+s+sd}{    \PYZdq{}\PYZdq{}\PYZdq{}}
             \PY{n}{pb} \PY{o}{=} \PY{n}{derivada\PYZus{}muda\PYZus{}pct}\PY{p}{(}\PY{n}{device}\PY{p}{[}\PY{l+s+s1}{\PYZsq{}}\PY{l+s+s1}{z\PYZus{}ang}\PY{l+s+s1}{\PYZsq{}}\PY{p}{]}\PY{o}{.}\PY{n}{values}\PY{p}{,} \PY{n}{device}\PY{p}{[}\PY{l+s+s1}{\PYZsq{}}\PY{l+s+s1}{v\PYZus{}ev}\PY{l+s+s1}{\PYZsq{}}\PY{p}{]}\PY{o}{.}\PY{n}{values}\PY{p}{)}
             \PY{n}{pa} \PY{o}{=} \PY{n}{N}\PY{o}{\PYZhy{}}\PY{l+m+mi}{1} \PY{o}{\PYZhy{}} \PY{n}{pb}
             \PY{n}{border\PYZus{}left} \PY{o}{=} \PY{n}{device}\PY{p}{[}\PY{l+s+s1}{\PYZsq{}}\PY{l+s+s1}{z\PYZus{}ang}\PY{l+s+s1}{\PYZsq{}}\PY{p}{]}\PY{o}{.}\PY{n}{values}\PY{p}{[}\PY{n}{pb}\PY{p}{]}
             \PY{n}{border\PYZus{}right} \PY{o}{=} \PY{n}{device}\PY{p}{[}\PY{l+s+s1}{\PYZsq{}}\PY{l+s+s1}{z\PYZus{}ang}\PY{l+s+s1}{\PYZsq{}}\PY{p}{]}\PY{o}{.}\PY{n}{values}\PY{p}{[}\PY{n}{pa}\PY{p}{]}
             \PY{n}{f\PYZus{}st\PYZus{}ev} \PY{o}{=} \PY{k}{lambda} \PY{n}{z}\PY{p}{:} \PY{o}{\PYZhy{}}\PY{p}{(}\PY{n}{z}\PY{o}{*}\PY{l+m+mf}{1e\PYZhy{}10}\PY{p}{)}\PY{o}{*}\PY{p}{(}\PY{n}{bias\PYZus{}j\PYZus{}m}\PY{p}{)}\PY{o}{/}\PY{n}{ev}
             
             \PY{k}{if} \PY{n}{z} \PY{o}{\PYZlt{}}\PY{o}{=} \PY{n}{border\PYZus{}left}\PY{p}{:}
                 \PY{k}{return} \PY{n}{f\PYZus{}st\PYZus{}ev}\PY{p}{(}\PY{n}{border\PYZus{}left}\PY{p}{)}
             \PY{k}{elif} \PY{n}{z} \PY{o}{\PYZgt{}}\PY{o}{=} \PY{n}{border\PYZus{}right}\PY{p}{:}
                 \PY{k}{return} \PY{n}{f\PYZus{}st\PYZus{}ev}\PY{p}{(}\PY{n}{border\PYZus{}right}\PY{p}{)}
             \PY{k}{return} \PY{n}{f\PYZus{}st\PYZus{}ev}\PY{p}{(}\PY{n}{z}\PY{p}{)}
         
         \PY{n}{device}\PY{p}{[}\PY{l+s+s1}{\PYZsq{}}\PY{l+s+s1}{bias\PYZus{}ev}\PY{l+s+s1}{\PYZsq{}}\PY{p}{]} \PY{o}{=} \PY{n}{device}\PY{p}{[}\PY{l+s+s1}{\PYZsq{}}\PY{l+s+s1}{z\PYZus{}ang}\PY{l+s+s1}{\PYZsq{}}\PY{p}{]}\PY{o}{.}\PY{n}{apply}\PY{p}{(}\PY{n}{constroi\PYZus{}bias}\PY{p}{)}
         \PY{n}{device}\PY{p}{[}\PY{l+s+s1}{\PYZsq{}}\PY{l+s+s1}{v\PYZus{}st\PYZus{}ev}\PY{l+s+s1}{\PYZsq{}}\PY{p}{]} \PY{o}{=} \PY{n}{device}\PY{p}{[}\PY{l+s+s1}{\PYZsq{}}\PY{l+s+s1}{v\PYZus{}ev}\PY{l+s+s1}{\PYZsq{}}\PY{p}{]}\PY{o}{+}\PY{n}{device}\PY{p}{[}\PY{l+s+s1}{\PYZsq{}}\PY{l+s+s1}{bias\PYZus{}ev}\PY{l+s+s1}{\PYZsq{}}\PY{p}{]}
         
         \PY{n}{dc} \PY{o}{=} \PY{n}{device}\PY{o}{.}\PY{n}{plot}\PY{p}{(}\PY{n}{x}\PY{o}{=}\PY{l+s+s1}{\PYZsq{}}\PY{l+s+s1}{z\PYZus{}ang}\PY{l+s+s1}{\PYZsq{}}\PY{p}{,} \PY{n}{y}\PY{o}{=}\PY{l+s+s1}{\PYZsq{}}\PY{l+s+s1}{v\PYZus{}st\PYZus{}ev}\PY{l+s+s1}{\PYZsq{}}\PY{p}{,} \PY{n}{grid}\PY{o}{=}\PY{k+kc}{True}\PY{p}{,} \PY{n}{title}\PY{o}{=}\PY{l+s+s1}{\PYZsq{}}\PY{l+s+s1}{Vst(eV) x z (Ang)}\PY{l+s+s1}{\PYZsq{}}\PY{p}{)}
\end{Verbatim}


    \begin{center}
    \adjustimage{max size={0.9\linewidth}{0.9\paperheight}}{output_9_0.pdf}
    \end{center}
    { \hspace*{\fill} \\}
    
    \hypertarget{calculando-autovaloresautovetores}{%
\subsection{Calculando
Autovalores/Autovetores}\label{calculando-autovaloresautovetores}}

Vamos utilizar o método da interação inversa para gerar os autovalores e
autovetores.


    % Add a bibliography block to the postdoc
    
    
    
    \end{document}
